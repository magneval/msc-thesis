\chapter{Introduction}

Integration of an office productivity suite into electronic document flows can be
divided into two parts: document management integration can be done alone, then
on top of that, adding workflow support to an office client is possible.

\section{Document management}
\label{sec:intro-docmgmt}

Every major software development project uses some kind of version control
system to track the source code of the project. The introduction of such a system
helps multiple developers to synchronize their work, helps debugging (when
retrieving older versions is necessary) and provides a way to document changes.

There are different approaches to version control:

\begin{itemize}
\item centralized
\item distributed
\item hybrid
\end{itemize}

Centralized (Subversion \cite{subversion} is probably the most
popular centralized solution) systems have a central server, distributed
systems operate in a distributed environment (Git \cite{git} is a good
example). A centralized system has a simpler architecture, but on the other
hand, it provides less features. Distributed systems can operate in more extreme
conditions, while they are harder to learn, because of their steep learning
curve.

The need for tracking versions and collaborating on documents is not specific to
software developers. An other type of similar software is usually referred to as
a document management system. It focuses on project management and it is meant to be
used by a wider range of users without specific software engineering knowledge.
That means they usually do not manage software source code files, but documents
created by office suites such as Microsoft Office \cite{mso} or
LibreOffice \cite{lo} (formerly OpenOffice.org \cite{ooo}).

There are numerous, currently unsolved problems in this area. Such enterprise
systems are supposed to be modular: each module communicates with the other
using an open (documented) protocol and companies are allowed to replace one
module with another implementation, as long as it communicates using the same
protocol. In fact, this is one of the major reasons for using open-source
software components in an enterprise environment: that way protocols are always
open, and companies can avoid vendor lock-in.

Enterprise companies using proprietary software, but wanting to adopt
open-source solutions can start the migration with server products. The benefit
of starting with servers is that only the central infrastructure has to be
modified. Given that the way of communication is unchanged, the clients can be left
untouched. Another approach is to focus on the client-side, as important data
is rarely stored there, so starting the migration with the client components
avoids the risk of a potential data loss, in case software quality is lower
than expected.

One of the most widely used document management servers is Microsoft
SharePoint \cite{sharepoint}. It is well integrated with Microsoft Office, but
previously it was not possible to replace the office suite with an open-source
alternative, because of the lack of SharePoint support. An open-source
alternative of SharePoint is Alfresco \cite{alfresco}, which provides similar
features, but uses its own, different protocol.

Finally, a radically different solution is a hybrid cloud-based approach: in that case
the client is a simple web browser and even the editing software is downloaded from
the server each time the user wants to edit the documents (Google
Docs \cite{google-docs} is a good example). While this method is easy to use, it
has its drawbacks as well. At the moment none of the cloud-based solutions have
all features a rich client such as Microsoft Office or LibreOffice provides.
Additionally, storing documents in the cloud requires a stable and fast network
connection during the editing process (which is currently isn't guaranteed in many
cases) and requires trust in the cloud provider when editing sensitive
documents of a company.

In the current thesis, we are presenting a centralized client-side solution for
using an existing, proprietary document management server with an open-source
office productivity suite. My solution is an extension to LibreOffice that adds
support for accessing and managing documents from a SharePoint server directly
from the office suite.

\section{Workflows}

Document management systems are often used as part of document-based workflows.
Workflows represent a collection of connected steps to achieve a certain goal.
A workflow engine can manage process definitions storing these sequences. A
process definition describes each step, and the connection between steps. The
steps are represented by nodes inside a process definition. Once the definition
is ready, it can instantiate a process definition, creating a process instance.
Such an object then always has a precise state:

\begin{itemize}
\item One or more of its nodes are active, and
\item It can have multiple associated variables.
\end{itemize}

This model then can help extract some of the business logic of applications
into more abstract process definitions, described in a more separated,
declarative way\footnote{Even if script tasks can contain imperative code, the
main structure of a process definition is always declarative.}.

It is a common pattern to use workflows to represent processes which describe a
set of changes of a given document. In short, they are referred to as
document-based workflows. Most such solutions are implemented with a strong
dependency between the document management server and the workflow engine,
which introduces the problem of vendor lock-in once again.

On the other hand, in case a decoupled document server / workflow engine
architecture is found to be necessary, the integration logic has to be implemented
in the client. Given that there is already a planned document management client
in the office suite running on the client, it's a natural extension to
introduce workflow integration in the office client to manage document-based
workflows.

Based on the experience from a real world use-case, we identified the following
requirements as mandatory:

\begin{itemize}
\item If a document is open, it's useful if operations on
that document can be done from the office suite, which already has a reference
to the document. Think about starting a chosen process definition to start a
new instance, associated with the current document.
\item The workflow server talks about tasks and group of tasks, and each task
is connected to a certain document. This is a conceptional difference to the model of
folders and files, which is already familiar to a user.
\item A common pattern is to assign a task to a group of users, and then
require exactly one member of the group to perform the task.
\item As outlined above, the ability to change the document server while not
touching the workflow engine -- and also the other way around -- may be a
requirement when planning a document-based workflow system.
\item It can be necessary to limit the write access to the associated document
when working on it to a certain part of the file. This helps the user
performing the task focusing on the work item itself, even if the document is a
large one.
\item It is uncomfortable if storing the document on the server and making a
decision as part of the current workflow task is not a single atomic operation.
\item It is important to realise that while finishing the work on a document
and marking a given workflow task is usually done together, sometimes only the
previous is wished. Support for such incremental snapshots is vital.
\item Finally, even if rolling back the workflow state is not trivial (given
that the workflow interacts with external services, rolling back the workflow
itself is not enough, external components should be notified using compensation
actions), support for releasing claimed group tasks is required.
\end{itemize}

There is one more operation where a modified workflow server is needed. In
general, but especially when group tasks or decisions are involved, creating
audit logs of process and node instances are important. Being able to them from the
office client makes viewing them much easier.

Based on that, we can build a list of functionality that can serve as a
motivation -- to help interacting with document-based workflows, in no
particular order:

\begin{itemize}
\item Workspaces, documents, task types and tasks should be represented to the
user following the well-known file-metaphor.
\item The document server should be decoupled from the workflow server, to
allow a modular architecture, having independent layers.
\item Support for both personal and indirectly assigned group tasks is required.
\item Even if group task is claimed by a user, the system should support
releasing the task back to the group, if the user realises he/she is not able to
perform it in time.
\item It should be possible to associate a new process instance to the current
document.
\item If the current task involves making decisions, the list of possible
choices should be retrieved from the workflow engine.
\item Accessing the audit log from the office client is required.
\item Supporting incremental snapshots till the task is marked as completed is
needed.
\item As a helper feature, masking the document so that only the relevant parts
are editable by the user is useful.
\end{itemize}

To summarize our goals:

\begin{itemize}
\item We want to design document management integration on the client-side.
\item We want to add workflow integration support to the solution.
\item We want to keep the created solution portable across different platforms.
\end{itemize}

The rest of this thesis is structured as follows. First, we introduce necessary
background knowledge, which was present before the current thesis, but is
needed to understand the rest of this work (Chapter~\ref{chap:background}). In
Chapter~\ref{chap:overview}, we propose our approach. Chapter~\ref{chap:design}
describes the design of the solution, as well as relation to the underlying
techniques. Next, we detail the implementation we created
(Chapter~\ref{chap:impl}) and also evaluate it (Chapter~\ref{chap:eval}).
Finally we present work related to the current solution
(Chapter~\ref{chap:related}) and give a summary, including future development
directions (Chapter~\ref{chap:conclusion}).
