\chapter{Introduction}

Every major software development project uses some kind of version control
system to track the source code of project. The introduction of such a system
helps multiple developers to synchronize their work, helps debugging, when
retrieving older versions is necessary and provides a way to document changes.

In fact, even if a version control system is not used, developers tend to
(poorly) invent a similar system implicitly: they copy files with a date string
appended a backup or manually copy files over to an other machine.

There are two main approaches for version control: centralized and distributed.
Centralized (Subversion\cite{subversion} is probably the most
popular centralized solution) systems have a central server, distributed
systems operate in a distributed environment (Git\cite{git} is a good such
example). A centralized system has a simpler architecture, but on the other
hand provides less features. Distributed systems can operate in more extreme
conditions, while it is harder to learn, because of the more complex
architecture.

The need for tracking versions and collaborating documents is not specific to
software developers. An other type of similar software is usually referred as
document management system. It focuses on project management: it is meant to be
used  not only by software developers, but users without technical knowledge.
That means usually it does not manage software source code files, but documents
created by office suites such as Microsoft Office\cite{mso} or
LibreOffice\cite{lo} (formerly OpenOffice.org\cite{ooo}).

There are numerous, currently unsolved problems in this area. Such enterprise
systems are supposed to be modular: each module communicates with the other
using an open (documented) protocol and companies are allowed to replace one
module with an other implementation, as long as it communicates using the same
protocol. In fact this is one of the major reasons for using open-source
software components in an enterprise environment: that way protocols are always
open, and companies can avoid vendor lock-in.

Enterprise companies using proprietary software, but wanting to adopt
open-source solutions can start the migration with server products. The benefit
of starting with servers is that only central infrastructure has to be
modified. Given that the communication works unchanged, the clients can be left
untouched. An other approach is to focus on the client-side, as important data
is rarely stored there, so starting the migration with the client components
avoids the risk of a potential data loss, in case software quality is lower
than expected.

One of the most widely used document management server is Microsoft
SharePoint\cite{sharepoint}. This is well integrated with Microsoft Office, but
previously it was not possible to replace the office suite with an open-source
alternative, because of the lack of SharePoint support. An open-source
alternative of SharePoint is Alfresco\cite{alfresco}, which provides similar
features, but uses its own, different protocol.

Finally a radically different solution is a cloud-based approach: in that case
the client is a simple web browser and even editing software is downloaded from
the server each time the user wants to edit the documents (Google
Docs\cite{google-docs} is a good example). While this method is easy to use, it
has its drawbacks as well. At the moment none of the cloud-based solutions have
all features a rich client such as Microsoft Office or LibreOffice provides.
Additionally, storing documents in the cloud requires stable and fast network
connection while editing documents (which is currently isn't guaranteed in many
cases) and requires trust from the cloud provider when editing sensitive
documents of a company.

In the current thesis, we are presenting a client-side solution for using an
existing, proprietary document management server with an open-source office
productivity suite. My solution is an extension to LibreOffice -- which is one
of the leading open-source office productivity suites -- that adds support for
accessing and managing documents from a SharePoint server directly from the
office suite.

The rest of this thesis is structured as follows. First, we introduce necessary
background knowledge, which was present before the current thesis, but is
needed to understand the rest of this work (Section 2). Then we continue with
giving an overview of the approach I'm proposing (Section 3). Section 4
describes the design of the solution, as well as relation with the underlying
techniques. Next, we detail the implementation we created (Section 5) and also
evaluate it (Section 6). Finally we present work related to the current
solution (Section 7) and give a summary, including future development
directions (Section 8).
