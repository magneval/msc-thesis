\section{Introduction}

Every major software development project uses some kind of version control
system to track the source code of project. This is useful for multiple
reasons:

\begin{itemize}
\item It helps multiple developers to synchronize their work.
\item It helps debugging, when retrieving older versions is necessary.
\item It provides a way to document changes.
\end{itemize}

In fact, even if a version control system is not used, developers tend to
(poorly) invent a similar system implicitly: they copy files with a date string
appended a backup or manually copy files over to an other machine.

There are many different types of version control software. One class of those
works in a centralized manner (Subversion\cite{subversion} is probably the most
popular centralized solution), and there are many other version control
projects which operate even in a distributed environment (Git\cite{git} is a
good such example). A centralized version obviously has a simpler architecture,
but on the other hand provides less features.

The need for tracking versions and collaborating documents is not specific to
software developers. An other type of similar software is usually referred as
document management system. It focuses on project management, and because of
that, it targets not only developers, but less technical users. That means
usually it manages not source code files, but documents created by office
suites such as Microsoft Office or LibreOffice.

There are numerous, currently unsolved problems in this area. Such enterprise
systems are supposed to be modular: each module communicates with the other
using an open (documented) protocol and companies are allowed to replace one
module with an other implementation, as long as it speaks the same
communication protocol. In fact this is one of the major reasons for using
open-source software components in an enterprise environment: that way
protocols are always open (even if poorly documented), and companies can avoid
vendor lock-in.

One approach here is to concentrate on server products: less technical users
don't see them anyway, so as long as the communication works unchanged, the
migration to open-source components on the server side is relatively easy. An
other approach is to focus on the client-side, as important data is rarely
stored there, so starting the migration with the client components avoids the
risk of a potential data loss, in case software quality is lower than expected.

In the current thesis, I am presenting a client-side solution for using an
existing, proprietary document management server with an open-source office
productivity suite. One of the currently most widely used document management
server is Microsoft Sharepoint. This is well integrated with Microsoft Office,
but previously it was not possible to replace the office suite with an
open-source alternative, because of the lack of Sharepoint support. My solution
is an extension to LibreOffice -- which is one of the leading open-source
office productivity suites -- that adds support for accessing and managing
documents from a Sharepoint server directly from the office suite.

The rest of this thesis is structured as follows. First, I introduce
necessary background knowledge, which was present before the current thesis,
but is needed to understand the rest of this work (Section 2). Then I continue
with giving an overview of the approach I'm proposing (Section 3). Section 4
describes the design of the solution, as well as relation with the underlying
techniques. Next, I detail the implementation I created (Section 5) and also
evaluate it (Section 6). Finally I present work related to the current solution
(Section 7) and give a summary, including future development directions
(Section 8).
