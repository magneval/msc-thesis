\chapter{Conclusion and Future Work}

\section{Document management}

We finished the design of the solution, and it was clear that creating a
LibreOffice extension that provides the necessary features is certainly
possible. We also implemented support for the most important use-cases.

As already discussed above, the most important weakness of the solution is that
it focuses only on the core functionality:

\begin{itemize}
\item As part of error handling, it presents permission errors to the user, but
changing permissions is not possible from the extension.
\item Managing users, tasks and links is yet to be designed and implemented.
\end{itemize}

Not ignoring the shortcomings, we can still conclude that the created solutions
makes migrating office productivity suites to open-source alternatives easier,
which was our initial goal.

There are some implementation and design issues that need to be addressed,
however, in the future:

\begin{itemize}
\item An implementation improvement would be to review the inherited codebase
and make sure that the user interface runs in a separate thread in all cases.
Currently there are cases when the user interface is not repainted due to
waiting for an input from the user.
\item Using LibreOffice's filepickers instead of reinventing our own would
result in a more consistent user experience.
\item Once CMIS will be widely implemented by proprietary document management
servers, it would make sense to use the CMIS protocol for client-server
communication instead of the native SharePoint one.
\end{itemize}

\section{Workflows}

Once the design of the document management part was ready, we continued
planning the workflow integration part. Our approach shows how a workflow
engine and an existing document-management server can be successfully
integrated to open-source office suites, addressing several use-cases
originating from an enterprise environment.

The following features are not designed in the current thesis, but they could
be done in the future:

\begin{itemize}
\item The gwt-console-server already provides REST API to access the audit log,
the gwt-console native UI could be improved to take advantage of that.
\item The current extension can communicate with jBPM only, it could be
extended to support various other pluggable workflow engines.
\end{itemize}

The following features are partly provided by our extension, but could be
improved further:

\begin{itemize}
\item Extending business decision support, adding integration for other kind of
decisions (see figures \ref{fig:decision-bpmn} and
\ref{fig:bpmn-gateway-types}) would be helpful.
\item A document can be attached to a single process instance only, this could
be extended to allow multiple process instances accessing the same document in
parallel -- but in that case proper locking should be designed first.
\end{itemize}

We hope that the release of our LibreOffice extension as an open-source
component contributes in general to the acceptance of open-source software in
enterprise environments.
