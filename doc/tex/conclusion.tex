\section{Conclusion and Future Work}

We finished the design of the solution, and it was clear that creating a
LibreOffice extension that has the required properties is certainly possible,
We also implemented support for the most important use-cases.

As already discussed above, the most important weakness of the solution is that
it focuses only on the core functionality:

\begin{itemize}
\item As part of error handling, it presents permission errors to the user, but
changing permissions is not possible from the extension.
\item Managing users, tasks and links is yet to be designed and implemented.
\end{itemize}

Not ignoring the shortcomings, we can still conclude that the created solutions
makes migrating office productivity suites to open-source alternatives easier,
which was our initial goal.

There are some implementation and design issues that need to be addressed,
however in the future:

\begin{itemize}
\item It was a design decision that we focused on document workspaces only,
however there are other areas to improve in LibreOffice for enterprise
environments, for example to design better integration with workflow management
-- when one node of a workflow completes once a user edited a complex document.
(Doing such an operation is usually not possible from a web browser if the
document is complex enough.)
\item An implementation improvement would be to review the inherited codebase
and make sure that the user interface runs in a separate thread in all cases.
Currently there are cases when the user interface is not repainted due to
waiting for an input from the user.
\item Using LibreOffice's filepickers instead of reinventing our own would
result in a more consistent user experience.
\item Once CMIS will be widely implemented by proprietary document management
servers, it would make sense to use the CMIS protocol for client-server
communication instead of the native SharePoint one.
\end{itemize}
