\chapter{Related Work}

\section{Document management}

We already saw that there is no solution today that is open-source, requires no
additional server-side installation and communicates with a Microsoft
SharePoint server (see \autoref{tab:related-cmp}).

However, there are similar projects, which -- even if they
solve a sightly different problem -- may provide excellent ideas to borrow.

\subsection*{OPAL}
We already introduced OPAL, which is open-source, but:

\begin{itemize}
\item Requires server-side modifications.
\item Works with an Alfresco server.
\item Not really maintained (does not work with latest stable OpenOffice.org).
\end{itemize}

However, its user interface and concepts are quite similar to our solution.

\subsection*{LibreOffice CMIS}
LibreOffice CMIS \cite{locmis} is a LibreOffice extension: a Java based
implementation of a Universal Content Provider for making any content in a CMIS
repository usable from LibreOffice. The main problem with it is that SharePoint
2007 does not implement CMIS, so this does not solve our problem at the moment.

\subsection*{SharePoint Connector}
The Oracle Connector for SharePoint Server \cite{oracle-sp-connector} is a
commercial OpenOffice.org extension, providing SharePoint support in
OpenOffice.org. It has multiple problems:

\begin{itemize}
\item It is not free. Sadly it is part of Oracle Open Office Enterprise
Edition, which is no longer available from the Oracle Store at the time of
writing (November 2011).
\item When we earlier checked it, It was only available on Windows 32-bit and it
required a server-side component as well.
\end{itemize}

\subsection*{libcmis}
libcmis \cite{libcmis} is a general purpose CMIS library, written in C++.
LibreOffice has a built-in Universal Content Provider built into its core,
internally using libcmis.

It has two issues at the moment:

\begin{itemize}
\item its CMIS support is incomplete
\item like LibreOffice CMIS, it does not support SharePoint 2007
\end{itemize}

The first problem is expected to be resolved in the long term, the second is
not a priority for this project.

\subsection*{Drupal SharePoint module}

An additional SharePoint module \cite{drupal-sp} for the Drupal content management system is also available. It solves a different problem, though:

\begin{itemize}
\item It can only read from a SharePoint server.
\item It is written for a CMS, not an office suite.
\item To reach its goals, it was enough to use the SOAP interface of the
SharePoint server, which is not enough for our purposes.
\end{itemize}

\begin{table}[H]
  \begin{center}
    \begin{tabular}{| l | l | l | l | l | l |}
    \hline
    \textbf{Feature} & \textbf{OPAL} & \textbf{LibO CMIS} & \textbf{libcmis} & \textbf{SP Conn.} & \textbf{Drupal} \\ \hline
    License          & open-source   & open-source        & open-source      & proprietary                   & open-source \\ \hline
    Server component & yes           & no                 & no               & yes                           & no \\ \hline
    \makecell[l]{SharePoint 2007 \\ support} & no & no    & no               & yes                           & partial \\ \hline
    \end{tabular}
  \end{center}
  \caption{Comparison of related document management works}
  \label{tab:related-cmp}
\end{table}

\section{Workflows}

\autoref{tab:related-wf-cmp} shows that there is no ready approach today for
document-based workflow management, that

\begin{itemize}
\item supports arbitrary process definitions
\item decouples document and workflow management
\item available under an open-source license
\end{itemize}

On the other hand, these are partly detailed in papers and implemented in some
other projects, which we present here.

\subsection*{Related papers}

Document-based workflows\footnote{Also known as document-centric or
document-driven workflows.} is an active research topic today. Multiple
interesting ideas are raised in recent papers.

\emph{A framework for document-driven workflow systems} \cite{paper-framework}:
\begin{itemize}
\item details why information and resource based workflows are also important, not just control based ones
\item use case: user changes the document, change listeners intercepts changes, check constraints, then accept or reject them
\item various complex features planned: split/merge of documents, different locking types
\item proposed implementation using SQL triggers
\item detailed comparison of control flow based versus document based approach
\end{itemize}

\emph{Mobility in the virtual office: a document-centric workflow approach} \cite{paper-mobility}:
\begin{itemize}
\item introduces decentralized document-driven workflows
\item changes to workflows travel with the documents
\item decentralization is handled with a peer-to-peer architecture
\end{itemize}

\emph{XDoC-WFMS -- A Framework for Document Centric Workflow Management System} \cite{paper-xdoc}
\begin{itemize}
\item presents use cases of intiutive workflow and document management integration: newspaper editing, processing job applications
\item proposed solution: documents have an embedded micro-agent, so the document itself will know where to go after a task is completed
\end{itemize}

\emph{Access control in document-centric workflow systems -- an agent-based approach} \cite{paper-access}
\begin{itemize}
\item this approach is without decoupling as well
\item the workflow object is proposed to communicate with the document object
\end{itemize}

\subsection*{SharePoint Designer}

The SharePoint Designer 2007 tool \cite{sp-designer} supports designing process definitions, to
be executed within the SharePoint document management server itself. It focuses on two features:

\begin{itemize}
\item triggers, executing an action when a document-related event occurs
\item a few builtin process definitions (review, approval, collecting signatures)
\end{itemize}

Its problems for our purposes:

\begin{itemize}
\item decoupling of the document and workflow server
\item document masking
\item standard workflow format (such as BPMN)
\end{itemize}

\subsection*{Liferay}

Liferay\footnote{The exact version I evaluated:
liferay-portal-tomcat-6.0.6-20110225.} \cite{liferay} is an open-source content management
system, focusing on the needs of enterprises. Regarding document management, it
comes with a \emph{Document Library} feature, which is similar to the one
Sharepoint and Alfresco provides.

The Document Library publishes the contents via WebDAV, so basic file
operations (open, save) are simple from an external application as well. It
also supports versioning, document metadata. Advanced Sharepoint-like actions
like commit message during checkin is not yet supported.

It has pluggable workflow integration, the following engines are supported out of the box:

\begin{itemize}
\item jBPM3
\item Kaleo
\end{itemize}

The later is configured by default, and it even has a few sample definitions
after installation. Sadly it seems Liferay invented its own schema for process
definitions when using Kaleo \cite{liferay-kaleo}, which resulted in not being
able to operate with standard BPMN or similar formats.

Its office integration is solely due to the WebDAV interface, with its known
limitations for our purpose.

\begin{table}[H]
  \begin{center}
    \begin{tabular}{| l | l | l | l |}
    \hline
    \textbf{Feature} & \textbf{Sharepoint Workflow} & \textbf{Liferay} & \textbf{jBPM} \\ \hline
    License          & proprietary                  & open-source      & open-source \\ \hline
    Decoupling       & no                           & no               & yes \\ \hline
    Office integration & yes                        & no               & no \\ \hline
    \makecell[l]{Standard process \\ definition format} & no & no      & yes \\ \hline
    \end{tabular}
  \end{center}
  \caption{Comparison of related workflow solutions}
  \label{tab:related-wf-cmp}
\end{table}

